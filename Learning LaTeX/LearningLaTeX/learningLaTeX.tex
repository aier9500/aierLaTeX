\documentclass[12pt]{article}

\usepackage{amsmath, amssymb, amsfonts} % Math symbols and operations
\usepackage{float}                      % Floating charts and images
\usepackage{enumerate}                  % Enumerated lists add-ons
\usepackage{ulem}                       % Underline functionalities
\usepackage{textcase}                   % Text cases functions
\usepackage{hyperref}                   % Hyperlinks

\title{Learning \LaTeX}
\author{Eugenio J. Wu}
\date{\today}

\pagestyle{empty}  % Turns off page numbers
\hbadness=10001    % Turns off \hbadness warnings
\parindent 0in     % Sets default paragraph indentation to 0


\begin{document}


\maketitle 
\begin{abstract}
    \noindent 
    Hello! This is my first \LaTeX\, document. \\[5pt]
    A retangle has side lengths of $(x + 1)$ and $(x + 3)$. The equation ${  A(x) = x^2 + 4x + 3  }$ gives the area of the rectangle. \\[5pt]    
\end{abstract} \pagebreak


\tableofcontents \pagebreak


\section{Math Expressions}
\subsection{Common Mathematical Notations}

Superscripts: \\[5pt]
$${  2x^2  }$$
$${  2x^n  }$$
$${  69x^{420}  }$$ \\[5pt]

Subscripts: \\[5pt]
$${  \pi_1  }$$
$${  f(1) = 0  }$$
$${  x_1 = 100  }$$ \\[5pt]

Trigonometric Functions: \\[5pt]
$${  \sin x' = \cos x  }$$
$${  \cos x' = -\sin x  }$$
$${  \tan ^2x = \sec ^2x - 1  }$$ \\[5pt]

Logarithmic Functions: \\[5pt]
$${  \log _b x = n  }$$
$${  \ln x' = \frac{1}{x}  }$$
$${  \int \frac{1}{x} = \ln x + C}$$ \\[5pt]

Roots: \\[5pt]
$${  \sqrt{4} = 2  }$$
$${  \sqrt[3]{27} = 3  }$$
$${  \sqrt[3]{x^2} = x^\frac{2}{3}  }$$
$${  \sqrt{  \sqrt[3]{x}  }  }$$

Fractions: \\[5pt]
About ${  \frac{2}{3}  }$ of the cup is full. \\[5pt]
About ${  \displaystyle \frac{2}{3}  }$ of the cup is full.
$${  \frac{1}{100}  }$$ \\[5pt]


\subsection{Brackets, Tables, and Arrays}

Brackets: \\[5pt]
(Normal parentheses, no need to escape), [Squared brackets, no need to escape], \{Curly brackets, you do need to escape\}, \textgreater You can type pointy brackets like with \textbackslash textgreater or \textbackslash textless \\[5pt]

Brackets in Math Mode: \\[5pt]
You can also write brackets in a more math-like aesthetic manner with functions \\[5pt]
$${ \left( \text{parenthesis} \right)}$$
$${ \left( \frac{99}{100} \right)}$$
$${ \left< \text{Dynamic pointy bracket} \right> }$$ \\[5pt]

Tables: \\[5pt]
\begin{center}
    \begin{tabular}{|c||c|c|c|}
        \hline
        Names & Jamal & Quandus & Luther \\ \hline
        Fried Chicken & 5 & 8 & 69 \\ \hline
        Watermelons & 1 & 2 & 5 \\ \hline
    \end{tabular} \vspace{20pt}
\end{center} \vspace{10pt}

\begin{table}[H]
    \centering
    \def\arraystretch{1.5} % Increasing vertical size of each row

    % Table 1: People and Food
    \begin{tabular}{|c||c|c|c|}
        \hline
        Names & Jamal & Quandus & Luther \\ \hline
        Fried Chicken & 5 & 8 & 69 \\ \hline
        Watermelons & 1 & 2 & 5 \\ \hline
    \end{tabular}
    \caption{People and Food}
    \vspace{20pt}

    % Table 2: People, Food, and Descriptions
    \begin{tabular}{|c||p{1in}|c|c|c|}
        \hline
        Names & The names of people participating in this chart & Jamal & Quandus & Luther \\ \hline
        Fried Chicken & Fried chicken is very tasty & 5 & 8 & 69 \\ \hline
        Watermelons & Watermelons are very juicy and sweet & 1 & 2 & 5 \\ \hline
    \end{tabular} 
    \caption{People, Food, and Descriptions}
    \vspace{20pt}
\end{table} \vspace{10pt}

Arrays: \\[5pt]
\begin{align} % When you use align, you are automatically in math display mode
    % Use ampersands to align
    &5x^2 \\
    &5x^2 \text{ is an equation}
\end{align}
\begin{align}
    5^2&=25 \\
    500&=500
\end{align} \vspace{10pt}


\section{Lists}

Lists: 
\begin{enumerate}
    \item Item 1
    \item Item 2
        \begin{enumerate}
            \item Item a
            \item Item b
            \begin{enumerate}
                \item Item i
            \end{enumerate}
        \end{enumerate}
\end{enumerate} \vspace{10pt}

\begin{itemize}
    \item Item 1
    \item Item 2
    \begin{itemize}
        \item Item a
        \item Item b
        \begin{itemize}
            \item Item i
        \end{itemize}
    \end{itemize}
\end{itemize} \vspace{10pt}

\begin{enumerate}[A.]
    \item Item A
    \item Item B
\end{enumerate} \vspace{10pt}

\begin{enumerate}[i)]
    \item Item i
    \item Item ii
\end{enumerate} \vspace{10pt}

\begin{enumerate} \setcounter{enumi}{2}
    \item Item 3
    \item Item 4
\end{enumerate} \vspace{10pt}


\section{Text Document Formatting}
\subsection{Text Formatting}

\begin{itemize}
    \item This will produce \textit{italicized} text.
    \item This will produce \textbf{boldfaced} text.
    \item This will produce \emph{emphasized} text. % Smartly formats text to create emphasis, usually, italics are used to emphasize text, however, since I'm using the ulem package here, italics are replaced with underlines
    \item This will produce \textsc{small caps} text.
    \item This will produce \texttt{monospace} text. 
    \item This will produce \MakeTextUppercase{uppercased} text.
    \item This will produce \MakeTextLowercase{LOWERCASED} text. 
    \item This will produce \uline{underlined} text. 
    \item This will produce \uuline{double underlined} text.
    \item This will produce \uwave{wavy-underlined} text. 
    \item This will produce \sout{strikethroughed} text. 
    \item This will produce \xout{hatched} text. 
    \item This will produce \dashuline{dash-underlined} text.
    \item This will produce \dotuline{dot-underlined} text. 
    \item This will produce a full link: \url{https://github.com/aier9500/aierNix}
    \item This will produce a hyperlink: \href{https://github.com/aier9500/aierNix}{my NixOS configs}.
\end{itemize} \vspace{10pt}

Different Sized Texts: 
\begin{itemize}
    \item The \begin{tiny}quick brown fox\end{tiny} jumps over the lazy dog. % tiny
    \item The \begin{scriptsize}quick brown fox\end{scriptsize} jumps over the lazy dog. % scriptsize
    \item The \begin{footnotesize}quick brown fox\end{footnotesize} jumps over the lazy dog. % footnotesize
    \item The \begin{small}quick brown fox\end{small} jumps over the lazy dog. % small
    \item The \begin{normalsize}quick brown fox\end{normalsize} jumps over the lazy dog. % normalsize
    \item The \begin{large}quick brown fox\end{large} jumps over the lazy dog. % large
    \item The \begin{Large}quick brown fox\end{Large} jumps over the lazy dog. % Large
    \item The \begin{LARGE}quick brown fox\end{LARGE} jumps over the lazy dog. % LARGE
    \item The \begin{huge}quick brown fox\end{huge} jumps over the lazy dog. % huge
    \item The \begin{Huge}quick brown fox\end{Huge} jumps over the lazy dog. % Huge
\end{itemize} \vspace{10pt}

Line Alignments: 
\begin{center}
    This line is centered. 
\end{center}

\begin{flushleft}
    This line is flushed left.
\end{flushleft}

\begin{flushright}
    This line is flushed right. 
\end{flushright}


\subsection{Document Formatting}

% Normally, sections, subsections, and subsubsections would be numbered so they are more differentiable from paragraphs; but to not mess up the numbering of this document, I am choosing to not number them
\section*{This is a Section}
Lorem ipsum dolor sit amet, consectetuer adipiscing elit. Aenean commodo ligula eget dolor. \\[5pt]

\subsection*{This is a Subsection}
Lorem ipsum dolor sit amet, consectetuer adipiscing elit. Aenean commodo ligula eget dolor. \\[5pt]

\subsubsection*{This is a SubsubSection}
Lorem ipsum dolor sit amet, consectetuer adipiscing elit. Aenean commodo ligula eget dolor. \\[5pt]

\paragraph*{This is a Paragraph}
Lorem ipsum dolor sit amet, consectetuer adipiscing elit. Aenean commodo ligula eget dolor. \\[5pt]

\subparagraph*{This is a subparagraph}
Lorem ipsum dolor sit amet, consectetuer adipiscing elit. Aenean commodo ligula eget dolor. \\[5pt]

\makebox[\linewidth]{\rule{\paperwidth}{1pt}}
\textasciicircum above is a pagewidth line \\[5pt]
\rule{\textwidth}{1pt}
\textasciicircum above is a textwidth line \\[5pt]
\begin{center}
    \rule{3in}{1pt}
\end{center}
\textasciicircum above is a 3 inch line \\[5pt]
\rule{\textwidth}{0.4pt}
\textasciicircum above is a thin (0.4pt) line \\[5pt]
\rule{\textwidth}{2pt}
\textasciicircum above is a thick(2pt) line \\[5pt]


\end{document}